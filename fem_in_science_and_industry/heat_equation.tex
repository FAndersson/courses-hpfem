\documentclass[a4paper,10pt]{article}
\usepackage[utf8]{inputenc}

\usepackage{amsmath}
\usepackage{amsfonts}
\usepackage{doi}

\begin{document}

\title{The heat equation}
\maketitle

\begin{abstract}
\noindent
This document deals with the basics of solving the heat equation using the
finite element method.
\end{abstract}

\section{Derivation of the heat equation}
Let $u$ be the temperature field in a solid $\Omega$. The heat flow $q$ in the
solid is given by
\[
q = -k \nabla u,
\]
where $k$ is the thermal conductivity in the solid. The change of temperature
in a volume $V \subset \Omega$ is given by the heat production inside the volume
minus the flow of heat out of the volume
\[
\frac{d}{dt} \int_V u \, dV
= \int_V f \, dV -\int_{\partial V} q \cdot n \, dS + \int
= \int_V f \, dV - \int_V \nabla \cdot q \, dV
= \int_V f + k \Delta u \, dV
\]
where we have assumed a homogeneous thermal conductivity in the last step. Since
this equation should hold for any $V$ we get
\begin{equation} \label{eq:heat_equation}
\frac{du}{dt} - \Delta u = f.
\end{equation}

\section{Finite element formulation}
First we discretize \eqref{eq:heat_equation} in time using the implicit Euler
method (assuming $f = 0$)
\begin{equation} \label{eq:heat_eqn_discretized_in_time}
\frac{u_{i + 1} - u_i}{\Delta t} - k \Delta u_{i + 1} = 0.
\end{equation}
Multiplying \eqref{eq:heat_eqn_discretized_in_time} with a test function $v$ and
integrating over $\Omega$ we get
\[
\int_{\Omega} \frac{u_{i + 1} - u_i}{\Delta t} \cdot v - k \Delta u_{i + 1} \cdot v \, dV = 0.
\]
Using Green's first identity and assuming that $v$ is zero on $\partial \Omega$
we have
\[
\int_{\Omega} (u_{i + 1} - u_i) \cdot v + k \Delta t \nabla u_{i + 1} \cdot \nabla v \, dV = 0.
\]

\end{document}