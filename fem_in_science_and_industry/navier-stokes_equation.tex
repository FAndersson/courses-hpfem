\documentclass[a4paper,10pt]{article}
\usepackage[utf8]{inputenc}

\usepackage{amsmath}
\usepackage{amsfonts}
\usepackage{doi}

\DeclareMathOperator{\tr}{tr}
\DeclareMathOperator{\sym}{sym}

\begin{document}

\title{The Navier-Stokes equation}
\maketitle

\begin{abstract}
\noindent
This document deals with the basics of solving the Navier-Stokes equation using
the finite element method.
\end{abstract}

\section{Derivation of the Navier-Stokes equation}
Let $u$ denote the velocity field in a fluid. From Cauchy's momentum equation we
have
\[
\frac{d}{dt} (\rho u) = \nabla \cdot \sigma + \rho g.
\]
Assuming $\rho$ is constant in time we get
\begin{equation} \label{eq:cauchys_momentum_equation}
\rho (\dot{u} + \nabla u \cdot u) = \nabla \cdot \sigma + \rho g,
\end{equation}
where $\dot{u} = \frac{\partial u}{\partial t}$. Now we use the Constitutive
model
\[
\sigma = -p I + d.
\]
Generally pressure is defined as
\[
p = -\frac{1}{3} \tr \sigma,
\]
so the tensor $d$ need to satisfy $\tr d = 0$. Assuming that $d$ is proportional
to $\varepsilon = \sym \nabla u$, and assuming an isotropic material we have
\[
d_{ij} = \lambda \delta_{ij} \varepsilon_{kk} + 2 \mu \varepsilon_{ij}.  
\]
Taking the $\tr d = 0$ equation into account we find $3 \lambda + 2 \mu = 0$,
i.e.
\[
d_{ij} = 2 \mu \left( \varepsilon_{ij} - \frac{1}{3} \delta_{ij} \varepsilon_{kk} \right).
\]
Plugging this into \eqref{eq:cauchys_momentum_equation} we get
\[
\rho (\dot{u} + \nabla u \cdot u) 
= -\nabla p 
+ \mu \left( \Delta u + \frac{1}{3} \nabla (\nabla \cdot u) \right) 
+ \rho g,
\]
which is Navier-Stokes equation. For an incompressible fluid we have
\[
\int_{\partial V} u \cdot n \, dS = \int_V \nabla \cdot u \, dV = 0,
\]
for any volume $V$ in the fluid, i.e. $\nabla \cdot u = 0$. So the
incompressible Navier-Stokes equation is
\[
\rho (\dot{u} + \nabla u \cdot u) 
= -\nabla p 
+ \mu \Delta u
+ \rho g,
\]
\[
\nabla \cdot u = 0.
\]
In the case of zero viscosity we get the Euler equations
\[
\rho (\dot{u} + \nabla u \cdot u) 
= -\nabla p 
+ \rho g,
\]
\[
\nabla \cdot u = 0.
\]

\end{document}